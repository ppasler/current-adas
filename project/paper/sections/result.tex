\label{chap:result}
In der vorgestellten Arbeit wurde eine Anwendung zur Müdigkeitserkennung entwickelt. Sie liest Rohdaten des Emotiv EEG-Headsets ein, verarbeitet und klassifiziert sie. Wird Müdigkeit erkannt, wird dies dem Fahrer mitgeteilt. 
Die Datenerhebung ist lose gekoppelt und kann die EEG-Daten über mehrere Wege übertragen. Für die Verarbeitung der Daten stehen mehrere Klassen zur Verfügung, die zu einer Verarbeitungskette verbunden werden. Für die Klassifizierung wurde ein KNN eingesetzt, welches performant die EEG-Sequenzen einteilt. Die Anwendung fügt sich in die bestehende Simulationsumgebung der Reutlingen University ein. Vier Testfahrten für Trainingsdaten wurden im Fahrsimulator der Hochschule Reutlingen samt Videobild aufgenommen.

Präzision und Genauigkeit (1) konnten nicht wie Erwartet umgesetzt werden. Die Erkennungsrate liegt derzeit bei ca. 61,5\% - das ist für ein sicherheitsrelevantes System deutlich zu niedrig. 
Fehlertoleranz (2) und Fehlerbehandlung (3) wurden umgesetzt und durch Unit- und Integrationstests abgesichert. 

Die Verarbeitungsgeschwindigkeit (4) der empfangen Daten beträgt  während eines Benchmarktests im Schnitt 200ms pro Sequenz und liegt damit deutlich unter der Liefergeschwindigkeit des EEGs (Sequenz $\equiv$ 128 Werte $\equiv$ 1000ms). 
Die Portierung des Systems ist theoretisch sehr einfach möglich, wenn während der Fahrt ein Laptop genutzt wird (5). Die Handhabung und Komfort des Headsets ist im Vergleich zu medizinischen EEGs deutlich verbessert (6). Das EEG lässt sich ähnlich wie eine Mütze tragen, jedoch ist die Einrichtung der Sensoren aufwändiger als erwartet (beste Signal-Qualität auf allen Sensoren). Die kabellose Übertragung sorgt für maximale Beweglichkeit, dennoch ist das Headset für den Produktiveinsatz ungeeignet.

Tests auf einem Smartphone oder Tablet müssen gesondert erfolgen, da es unter anderem von den Treibern des EEG Headsets abhängt (7). Lasttests wurden nur auf dem Entwickler Laptop (Lenovo ThinkPad W530) durchgeführt (8). Der Einbau in die Simulationsumgebung ist bis zum Eintragen der Werte im CAN-Bus umgesetzt. Das Herausnehmen der Werte ist noch nicht vollständig umgesetzt, da bisher die Integration der CarInterface Anwendung nicht funktionierte (9). Die PoSDBoS Anwendung lässt sich sehr gut auf verteilten Systemen ausführen, auch über den Anwendungsfall des Fahrsimulators hinaus. Die akquirierten Daten können via http an die Verarbeitungsschicht übertragen werden. Auch das Verschicken des Ergebnisses der Klassifizierung per http wäre denkbar und einfach umzusetzen (10). Die Art der Benachrichtigung über eine erkannte Müdigkeit ist noch nicht vollständig umgesetzt. Derzeit wird dem Fahrer entweder ein grüner oder roter Bildschirm angezeigt.

Die komplette Anwendung ist mit Docstring\footnote{\url{https://www.python.org/dev/peps/pep-0257/\#what-is-a-docstring}} versehen, aus denen eine html-Dokumentation erzeugt werden kann. Weiterhin sind schwierige Code-Teile mit einfachen Beispielen erweitert, um den Einstieg zu erleichtern. Ob es so möglich ist, als Neuling, selbständig die Anwendung zu verstehen bzw. zu erweitern hängt wohl auch von den jeweiligen Grundkenntnissen ab. Werden jedoch Veränderungen vorgenommen, kann durch die Unit-Tests sichergestellt werden, dass die Klassen weiterhin das Richtige tun. Integrationstests sind nur teilweise umgesetzt.