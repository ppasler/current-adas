\label{sec:classification}
Nachdem in vorherigen Abschnitt eine passende Merkmalsmenge erstellt wurde, geht es nun um die Entscheidung, ob der Fahrer Müde oder Wach ist bzw. ob das System eine Müdigkeitsmeldung erscheinen lässt. Für diese Klassifizierung werden im allgemeinen Machine-Learning-Algorithmen verwendet. Anhand von markierten Datensätzen wird versucht den Algorithmus zu Trainieren (Überwachtes Lernen). Dies dient dem Ziel, dass er auch unbekannte Daten klassifizieren kann. Dieser Vorgang wird Generalisierung bezeichnet und ist auch im menschlichen Lernen ein wichtiger Schritt.

Für die Anwendung wurde zur Klassifizierung ein künstliches Neuronales Netz (KNN) ausgewählt. Es basiert auf einem erweiterten Perceptron / McCulloch-Pitts-Neuron \cite{ann} und ist der Funktionsweise des menschlichen Gehirns bzw. seinen Neuronen nachempfunden\cite{marsland_opac-b1129336}. Ein KNN lässt sich im einfachste Fall durch eine Merkmalsmenge $X = x_1, x_2 ... x_n$, dazugehörige Gewichte $W = w_1, w_2 ... w_n$, eine Übertragungsfunktion $\sum$ und eine Schwellwertfunktion $\theta$ beschreiben (Abb. \ref{fig:perceptron}).

\begin{figure}[h] 
  \begin{center}
    \includegraphics[width=5.5cm]{perceptron}
    \caption[Schema eines Perceptrons / McCulloch-Pitts-Neurons]{Darstellung eines McCulloch-Pitts-Neurons. Die Merkmale $X$ werden mit den Gewichten $W$ multiplziert und in $\sum$ summiert. Wenn $h > \theta$ "`feuert"' das Neuron ($o = 1$) \cite{marsland_opac-b1129336}. \label{fig:perceptron}}
  \end{center}
\end{figure}

Dieser Aufbau kann schon einfach Aufgaben, wie bspw. ein logisches "`UND"', lösen. Jedoch lässt sich schon ein logisches "`XOR"' nicht mehr abbilden. Dafür müssen weitere Schichten von Neuronen (Hidden Layers) hintereinander geschaltet werden - das sog. Multi Layer Perceptron (MPL, Abb. \ref{fig:mlp}).

\begin{figure}[h] 
  \begin{center}
    \includegraphics[width=5.5cm]{mlp}
    \caption[Schema eines Multi-Layer-Perceptrons]{Darstellung eines Neuronalen Netzes mit mehreren Schichten (Multi Layer Perceptron, MLP)\cite{marsland_opac-b1129336}. \label{fig:mlp}}
  \end{center}
\end{figure}

Es exisitiert kein bekannter Algorithmus für die Wahl der optimalen initialen Parameter eines KNNs. Vuckovic et al. \cite{Vuckovic2002349} hatten sich mit diesem Thema genauer beschäftigt und die Ergebnisse werden für die Versuche herangezogen.
\textbf{TODO}