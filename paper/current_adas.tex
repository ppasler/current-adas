\documentclass[8pt,a5paper]{acm_proc_article-sp}
% fix umlauts
\usepackage[ngerman]{babel}
\usepackage[utf8]{inputenc} 
\usepackage[T1]{fontenc}  % Times new Roman
\usepackage{mathptmx}
\usepackage[]{blindtext}
\usepackage[a5paper, left=1.70cm, right=1.20cm, top=1.30cm, bottom=1.40cm]{geometry}
\usepackage{epstopdf}
% balance columns. original from acm doesn't work
\usepackage{balance}
% colors
\usepackage[compact]{titlesec}
\usepackage{natbib}
\titlespacing{\section}{0pt}{*0.7}{*0.5}

\usepackage[nolist]{acronym}
 
\usepackage[usenames,dvipsnames]{xcolor}
% automatic crosslinks
\usepackage[colorlinks=true,citecolor=black,linkcolor=black,filecolor=black,urlcolor=black,pagebackref=true]{hyperref}
\hypersetup{colorlinks=true,citecolor=black,linkcolor=black,filecolor=black,urlcolor=black,pagebackref=true}
%glossaries
%\usepackage{makeidx}
%\makeindex
%\usepackage[nomain]{glossaries}
%\makeglossaries
%\makeindex

\newcommand{\glspl}[1]{{#1}}

% http://en.wikibooks.org/wiki/LaTeX/Glossary
% http://mirror.informatik.uni-mannheim.de/pub/mirrors/tex-archive/macros/latex/contrib/glossaries/glossariesbegin.pdf

% Definitionsliste
\newcommand{\defitem}[1]{\item[#1]\phantomsection\label{#1}\hfill\\} 
\newcommand{\defref}[1]{\hyperref[#1]{#1}} 
\newcommand{\rem}[1]{}
% Marking colors
\definecolor{todo}{rgb}{1,0.2,0.2}
\definecolor{reconsider}{rgb}{0.6,0.6,0.3}
\newcommand{\todo}[1]{{\color{todo} #1}}
\newcommand{\reconsider}[1]{{\color{reconsider} #1}}



\graphicspath{{images/}{./}}

\title{Fahrerassistenzsysteme: Müdigkeitserkennung im Simulationsumfeld \titlenote{ \scriptsize \flushleft Betreuer Hochschule:  \  \ Prof. Dr. Martinez\\ \qquad \qquad \qquad \qquad \quad \ \  Hochschule Reutlingen\\ \qquad \quad \quad \quad \qquad \qquad \ \ Natividad.Martinez@Reutlingen-\\ \qquad \qquad \qquad \qquad \quad \ \ University.de\\  Wissenschaftliche Vertiefungskonferenz 2015\\ Wissenschaftliche Vertiefungskonferenz \\ 18. November 2015, Hochschule Reutlingen\\  \copyright 2015 Paul Pasler}}


\numberofauthors{3}
\author{
	\alignauthor
	  \center
		\aufnt{Paul Pasler}\\
          \affaddr{Reutlingen University}\\
        \textbf{\textsf{Paul.Pasler@Student.Reutlingen-University.DE}}
}



\begin{document}

\begin{acronym}
\acro{FAS}{Fahrerassistenzsystem}
\acro{FASs}{Fahrerassistenzsysteme}
\acro{ME}{Müdigkeitserkennung}
\acro{MESs}{Müdigkeitserkennungssysteme}
\acro{ADAS}{Advanced Driver Assistance System}
\acro{ADASs}{Advanced Driver Assistance Systems}
\acro{bspw}{beispielsweise}
\acro{RTU}{Reutlingen University}
\acro{BS}{Body-Sensoren}

\end{acronym}


\maketitle
\sloppypar{
\begin{abstract}
Mit Fortschreiten der Technik, verbreiten sich \acl{FAS}en immer weiter. Besonders der Teilbereich der \acl{ME} hilft schwere Unfälle zu vermeiden. Die \acl{ME} mit Body-Sensorik liefert sehr gute Ergebnisse, scheitert aber in der Praxis häufig auf Grund seines invasiven Charakters. Im Simulationsumfeld der \acl{RTU} wird ein Konzept entwickelt, dass Körperfunktionen überwacht und diese auswertet, ohne den Fahrer zu beeinträchtigen. Weiterhin wird die Möglichkeit einer einfachen Portierung der Anwendung vom Simulator in ein echtes Fahrzeug geprüft. Das vorgestellte Konzept, soll somit ein Höchstmaß an Genauigkeit, Tragekomfort und Mobilität vereinen.

\end{abstract}

\keywords{
\acf{ADAS}, \acl{FAS}, \acl{ME}
}

\category{A.0}{ACM}{
Experimentation
}

\section{Einleitung}
\label{chap:intro}
Fristeten \acl{FASs} vor wenigen Jahren ein Nischendasein in Oberklassewagen, werden sie immer günstiger und beliebter. So halten sie auch in Mittel- und Kleinwagen Einzug und helfen bei der Vermeidung schwerer Unfälle. Die \acl{ME} unterstützt dem Fahrer und rät ihm zu gegebenen Anlass eine Pause einzulegen. Hierbei zeigt die Erkennung mit \acl{BS} (EEG, EKG) sehr gute Ergebnisse. Der Tragekomfort macht diese Systeme aber für die Praxis meist ungeeignet. \\

Ziel der Arbeit ist es, ein Konzept zu entwickeln, dass das Problem der invasiven Sensoren weitestgehend eliminiert und den Fahrer wenig bis gar nicht beeinträchtigen. Dies soll im Simulationsumfeld der \acl{RTU} getestet werden und später in einem echten Fahrzeug validiert werden. Weiterhin soll das System in der Lage sein mit anderen Systemen gekoppelt zu werden, um die Ergebnisse zu verbessern.\\ 

Die Ausarbeitung gliedert sich folgendermaßen. Im Kapitel \ref{chap:me} werden verschiedene  Umsetzungsvorschläge zur \acl{ME} aufgezeigt und verglichen. Das Konzept eines portablen Systems zur \acl{ME} mit \acl{BS} wird im Kapitel \ref{chap:prop} vorgestellt. Der Versuchsaufbau und das Testszenario im Simulationsumfeld der \acl{RTU} ist Thema von Kapitel \ref{chap:eval}. Das Ergebnis und weitere Schritte werden in Kapitel \ref{chap:result} und \ref{chap:outro} beschrieben.\\

\subsection{\acl{FASs}}
- Studie zu Unfällen -
 \cite{Feld:2010:MIA:1719970.1720063}

\acl{FASs} erhöhen dem Fahrer zum einen den Komfort und / oder die Sicherheit beim Fahren. So führen Einparkassistent,  Geschwindigkeitsregelanlage oder Navigation zu einer deutlich entspannteren Fahrt. Spurhalte-, Spurwechsel- oder Notbremsassistent wiederum unterstützen bei potentiell gefährlichen Manövern. Auch die \acl{ME} fällt in die zweite Kategorie (mehr dazu in Kapitel \ref{chap:me}).\\

Kompaß \cite{fasFuture} unterteilt \acl{FASs}, gemessen an der Reaktionszeit, in Planung, Führung und Stabilisierung. Hierbei fällt \acl{bspw} Navigation in die Planungsebene, da die Berechnung der Route mit unter mehrere Minuten brauchen kann. Auf Führungsebene werden dem Fahrer Empfehlungen und Warnungen innerhalb weniger Sekunden mitgeteilt, auf die er dann reagieren kann. Greift das System selbständig in den Fahrprozess ein, muss dies meist innerhalb von Millisekunden geschehen und dient oftmals zur Stabilisierungen, wie \acl{bspw} bei einem Fahrdynamik-Regelsystem.\\

Ein \acl{FAS} kann auf verschiedenste Arten mit dem Fahrer kommunizieren. Es handelt sich um eine klassische HCI-Schnittstelle. Am gebräuchlichsten, auch für sonstige Warnungen, sind schon seit längerem Optische und Akustische Signale. Aber auch Vibrationen in Lenkrad und Sitz zeigen gute Ergebnisse, wenn zwischen Signal und Nachricht ein Zusammenhang besteht (Bspw. Vibriert das Lenkrad bei verlassen der Spur).
\cite{Bertoldi:2010:MAD:2002368.2002370} beschreibt hierzu die verschiedenen Anwendungsgebiete und Unterschiede. \\

Jeder Automobilhersteller entwickelt mittlerweile seine eigenen \acl{FASs}. Datenerhebung (Sensoren), Berechnung und Kommunikation werden vom Fahrzeug selbst durchgeführt. Durch die Abschottung des Fahrzeugs sind Fahrzeugdaten nicht öffentlich zugänglich und können nur schwer von Außenstehenden genutzt werden. 

Für wissenschaftliche Arbeiten bleibt entweder eine Kooperation mit Automobilherstellern oder das Ausweichen auf andere Devices, wie ein Smartphone und die Nutzung von Daten aus dem Internet (\acl{bspw} Kartendienste). Chen \cite{Chen:2015:ISV:2742647.2742659} und You \cite{You:2013:CAA:2462456.2465428} verfolgten diesen Ansatz. Smartphone bieten durch ihren hohen Verbreitungsgrad eine günstige Alternative zu eingebauten Systemen, können jedoch nicht auf Daten des Fahrzeugs zugreifen und müssen einfache Daten, wie \acl{bspw} Geschwindigkeit, selbst berechnen.\\



\subsection{\acl{BS}}
\acl{BS} messen verschiedenste Werte eines lebenden Körpers, wie den Puls, Temperatur oder Hirnwellen. Meistens werden sie direkt am oder im Körper eingesetzt.\\

Bei der Elektroenzephalografie (EEG) werden Elektroden auf der Kopfhaut angebracht und damit die Aktivität des Gehirns gemessen. Sie wird in der Medizin für die Diagnose von Epilepsie oder bei Komapatienten eingesetzt. Zudem findet sie in Schlaflabors Anwendung, um verschiedene Schlafphasen zu erfassen. Der Zusammenhang von Schlaf und Hirnaktivität kann auch bei der \acl{ME} in Fahrzeugen genutzt werden, um \acl{bspw} ein drohenden Sekundenschlaf zu erkennen \cite{Santamaria_eeg}. \\

Das Elektrokardiogramm (EKG) misst die Herzspannungskurve und stellt die Aktivität des Herzmuskels dar. So lassen sich vielfältige Aussagen über den Zustand des Herzens machen. Weiterhin können Herzrhythmus und -frequenz Hinweise auf eine einfallende Müdigkeit des Fahrers geben. Für die Messung des EKG Signals existieren mehrere Methoden, im medizinischen Bereich werden Elektroden an verschiedenen Körperstellen geklebt. Weiterhin werden, vor allem im Sport, Sensoren mit Gummibändern an Brust oder Handgelenk befestigt. \\

Eine weitere Möglichkeit Körperfunktionen aufzuzeichnen ist die Elektrookulografie (EOG). Hierbei kann die Bewegung der Augen bzw. das Ruhepotential der Netzhaut gemessen werden. Dazu werden Elektroden entweder rechts und links oder oben und unter dem Auge angebracht.\\

Für das Portable System zur Müdigkeitserkennung wird ein EEG Epoc Emotiv \footnote{\url{https://emotiv.com/epoc.php}} und ein Brustband EKG Zephyr Bioharnes \footnote{\url{http://www.zephyranywhere.com/products/bioharness-3}} eingesetzt.


\subsection{Simulationsumgebung}
\label{subsec:sim}
Die Simulationsumgebung der \acl{RTU} ermöglicht es, Testfahrten unter möglichst realistischen Bedingungen zu durchzuführen (Siehe Abb. \ref{fig:architecure}). 
Sie ist ausgestattet mit einem Autositz, einem Lenkrad und Pedalen, sowie einer Gangschaltung. Weiterhin bringen drei $48^{\prime\prime}$ Monitore und ein Dolby-Surround-System einen audiovisuellen Eindruck einer Fahrsituation.

Auf technischer Ebene wird das System von drei Computer mit Leben gefüllt. Auf dem Simulations-Computer läuft die vom DFKI Saarbrücken entwickelte Software OpenDS \footnote{\url{http://www.dfki.de/web/aktuelles/aktuelles/cebit2013/opends}}. Hier können mehrere Karten und Konfigurationen eingestellt und getestet werden. Alle Simulationsdaten, werden via TCP/IP an den Daten-Computer gesendet. Dieser stellt die Daten über ein CAN-Interface zur Verfügung und loggt diese zusätzlich. Das CAN-Interface simuliert die Kommunikation mit dem Steuergerät eines realen Fahrzeugs. Der Anwendungs-Computer kann die CAN Daten über eine Schnittstelle empfangen und seine eigentliche Arbeit verrichten. Für Ein- und Ausgaben der Anwendung, befindet sich ein Touchscreen neben dem Lenkrad.
\begin{figure}[h]Weiterhin 
  \begin{center}
    \includegraphics[width=6.5cm]{img/architecture}
    \label{fig:architecure}
    \caption[Aufbau des Simulators]{Der Aufbau des Simulators der \acl{RTU}}
  \end{center}
\end{figure}



\section{Systeme zur \acl{ME}}
\label{chap:me}
Müdigkeit senkt die Konzentrationsfähigkeit des Fahrers, kann zu einer erhöhten Reaktionszeit und Fehleinschätzungen führen. So stellt es, unter anderem, der Deutscher Verkehrssicherheitsrat in einem Beschluss von 2009 \cite{DVR:Online} fest. Ursachen können wenig Schlaf, lange Fahrzeiten, Medikamente oder Alkohol sein.
Die \acl{ME} versucht an Hand verschiedener Daten und Sensoren, frühzeitig zu erkennen, ob der Fahrer gerade Anzeichen einer bevorstehenden Müdigkeit zeigt und empfiehlt eine Pause. Dabei soll nicht nur während eines Micro- oder Sekundenschlafs, sondern schon früher gewarnt werden. \\

\subsection{Überblick und Klassifizierung}
Die Erkennung von Müdigkeit kann auf ganz verschiedene Arten gelöst werden. Ein Ansatz versucht über Körpersignale herauszufinden, ob eine Müdigkeit bevorsteht. Wohingegen mit der Analyse des Fahrverhaltens das Selbe mit Sensoren an und im Auto realisiert wird.
Bei der Erkennung über Körpersignale, können wiederum \acl{BS} oder Computer-Vision-Techniken zur Überwachung des Fahrers genutzt werden. Zu unterscheiden ist weiterhin die physische und psychische Müdigkeit, welche sich jedoch beide negativ auf die Fähigkeiten des Fahrers auswirken. Alle Verfahren, die auf Senoren am Körper, die extra angezogen werden (bspw. ein Pulsmesser am Ohr, EEG) werden als inversive Verfahren bezeichnet.Weiterhin 

Allen System gemein ist die Nutzung von Klassifizierungs- bzw. Machine-Learning-Algorithmen. Die gesammelten Daten geben nur Hinweise und sind kein Garant für eine Erkennung von Müdigkeit. \acl{MESs} wandeln hier auf einem schmalen Grad, da es zum einen um die Verhinderung schwerer Unfälle geht, zum anderen aber ein falsch auslösendes System die Akzeptanz vermindert und im schlimmsten Fall zu einer Deaktivierung führt. Um falsche Erkennungen weiter zu minimieren, werden oftmals mehrere Ansätze kombiniert.

In der Praxis setzen Automobilhersteller wie Daimler \cite{Daimler} und Volkswagen, sowie Automobilzulieferer wie Bosch \cite{Bosch} auf die Analyse des Fahrverhaltens. Insbesondere Spurhalten und ruckartiges Gegenlenken scheinen ein signifikantes Indiz für beginnende Übermüdung zu sein. Weiterhin sind externe Geräte und einige Apps für Smartphones erhältlich. \\

\subsection{Stand der Technik}
Wie bereits erwähnt, existieren verschiedene Vorgehen, für die \acl{ME} im Fahrzeug. Im folgenden werden verschiedene Forschungsergebnisse vorgestellt und bewertet, eine weitere Übersicht findet sich auch in \cite{Sahayadhas_121216937}.\\

Es existieren einige Ansätze die mit Hilfe von Kameras den Fahrer und die Straße beobachten. Zhang et al. \cite{Zhang:2015:RSD:2753829.2629482} stellen hierzu eine Applikation mit der Verbindung eines Farb- und Tiefenbildes vor. Mit Hilfe einer Microsoft Kinect werden sowohl die Kopfpose, als auch die Augenstatus bestimmt. Um das System robuster zu gestalten, wird aus dem Farbbild,  zusätzlich zum Vorhandenen, das Tiefenbild berechnet. Mit der CarSafe App entwickleten You et al. \cite{You:2013:CAA:2462456.2465428} ein visuelles System zur Überwachung des Fahrers und der Straße. Hierfür genügt ein aktuelles Smartphone. Die App deckt hierbei neben der \acl{ME} auch andere Gefahrensituation (\acl{bspw} zu dicht Auffahren) ab. Es werden eine Analyse des Fahrers (Kopfpose und Augenstatus), sowie der Fahrweise kombiniert und der Fahrer gewarnt. Kamerabasierte Systeme sind angenehm für den Fahrer, da er keine weitere Hardware (Sensoren) installieren muss. Jedoch ist eine Kamera optischen Grenzen unterworfen, was den Einsatz bei Nacht oder schlechtem Wetter erschwert. Für eine \acl{ME} mit Smartphone, aber ohne Kameraeinsatz, könnte \acl{bspw} die App V-Sense \cite{Chen:2015:ISV:2742647.2742659} genutzt werden, da sie lediglich  eingebaute Sensoren nutzt.

Bundele and Banerjee \citep{Bundele:2009:DFV:1806338.1806478} zeigten, dass Müdigkeit über die elektrodermale Hautreaktion und Pulsoxymetrie erkannt werden kann. Die elektrodermale Hautreaktion (galvanische Hautreaktion, GSR) misst hierbei die Hautleitfähigkeit und hängt mit der Schweißproduktion zusammen. Bei der Pulsoxymetrie kann, durch ein optische Verfahren, die Sauerstoffsättigung des Blutes gemessen werden. In diesem Fall bedeutet eine geringere Sättigung ein erhöhtes Müdigkeitsgefühl. Diese Werte werden durch \acl{BS} ermittelt und werden mit einem Multi Layer Perceptron (MLP) klassifiziert. Interessant ist zudem der Einsatz von sogenannten Smart-Clothes (E-textiles), welche die Sensoren in der Kleidung eingearbeitet haben und somit zu ein Non-Inversiven Ansatz führen.\\

Park et al. \cite{Park:2009:DDD:1667780.1667798} beschränken sich in ihrer Arbeit auf die Analyse der Pulswelle durch Photoplethysmography (PPG), mit einem eingebauten Sensor am Lenkrad. Dies stellt schon ein größeren Eingriff in die Umgebung des Fahrzeugs dar, als es im Abschnitt zuvor der Fall war. Die Daten der PPG werden mit einer Support Vector Maschine (SVM) eingeordnet. Es zeigte sich, dass die Ausschlagshöhe des Puls ein gutes Mittel für die Erkennung von Müdigkeit darstellt. Um die Ergebnisse zu verbessern, wurde die entwickelte Software mit einem zuvor entwickelten visuellen System zur Kopfbewegung gekoppelt. \\

Zhang et al. \cite{zhang_6513058} befassten sich ebenfalls mit der Überwachung von Herzfunktionen, jedoch mit Hilfe eines EKGs. Sie fanden heraus, dass die Wavelet Packet Energie in bestimmten Frequenzbereichen auf eine Veränderung des QRS hinweist, welcher wiederum als Indiz für einfallende Müdigkeit genutzt werden kann. Dies erhöht nach ihren Angaben die Geschwindigkeit und die Genauigkeit der Erkennung. In Verbindung mit der Wavelet Entropie kann eine trainierte SVM nahezu 100\% erreichen. Rogado et al. \cite{Rogado_4913155} nutzen ebenfalls Daten aus dem EKG, um daraus die Herzfrequenzvariabilität (HFR, english HRV) zu berechnen und drohendes Einschlafen zu erkennen. Ähnliches Verhalten untersuchten Vicente et al. \cite{Vicente_6164509} \textbf{TODO}.  

EEG \\
Subasi \cite{Subasi:2005:ARA:1707423.1707550}\\
Wilson \cite{wilson_890161}\\
Khalifa \cite{khalifa_893852}\\
Vuckvic \cite{Vuckovic2002349}\\
Huang \cite{Huang_548971}\\
Johnson et. al \cite{Johnson11} nutzen EEG.\\
Lin et al. \cite{Lin05eeg-baseddrowsiness}
Ronzhina et al. \cite{Ronzhina:2011:UEV:2093698.2093733} evaluierten den Flicker-Fusion Test \\


\subsection{Vergleich}
Systeme die das Fahrverhalten analysieren und daraus Rückschlüsse auf den Wachheitsgrad des Fahrers ziehen, sind in der Praxis weit verbreitet. Die Sensoren sind im Fahrzeug fest verbaut und das System kann auf das jeweilige Fahrzeug abgestimmt werden. Hier liegt schon das erste Problem, da die Systeme nicht in einem Fahrzeug einer anderen Marke einsetzbar sind. Weiterhin finden sich im Internet viele Fragen zum Grund der Pausenempfehlung. \\

Videobasierte Systeme erfüllen ebenfalls ihren Zweck, sind aber leicht durch äußere Einflüsse, wie \acl{bspw} schlecht Lichtverhältnisse, beeinflussbar. Auch wenn Systeme mit Infrarotkameras diese Probleme vermindern, kann es sein, das \acl{bspw} Augen von (Sonnen-)Brillenträgern nicht richtig erkannt werden können. \\

Die meisten \acl{BS} liefern sehr gute Ergebnisse, sind aber wegen ihrer invasiven Eingenschaften (Die Sensoren liegen direkt am Körper an) weniger für den Serienbetrieb geeignet. Um dieses Problem zu lösen, stellte der französische Automobilzulieferer seinen intelligenten Autositz "Active-Wellness" vor. Der Sitz ist mit passiven Sensoren ausgestattet und misst ständig den Herzrhythmus, die Atmung und weitere biometrische Daten, um bei einfallender Müdigkeit gegenzusteuern \footnote{\url{http://www.faurecia.de/node/1780}}. Auch Pulsmesser am Handgelenk werden immer besser, sind dabei nicht größer als eine Uhr und bieten ein vertretbaren Tragekomfort. \\

Aufgrund der hohen Genauigkeit von EEG, EKG und Co. sind diese Systeme zur Verbesserung oder Validierung anderer Systeme in der Testphase nützlich. Gelingt es zudem den invasiven Charakter gering zu halten oder gar ganz zu vermeiden, sind Systeme mit \acl{BS} auch für den Produktiveinsatz geeignet. Es bleibt zu zeigen, dass Genauigkeit und Tragekomfort im richtigen Verhältnis stehen. Hat sich ein System in der Simulationsumgebung bewährt, sollte das System in einem realen Fahrzeug getestet werden können.  \\

\begin{itemize}
  \item Gegenüberstellung der verschiedenen Ansätze
  \item Wo gibt es schwächen? Welche Nische wurde noch nicht beleuchtet
\end{itemize}

\section{Portables System zur \acl{ME} mit \acl{BS}}
\label{chap:prop}
Wie im vorherigen Absatz gesehen, existieren sehr viele verschiedene Lösungen zur \acl{ME} in Fahrzeugen. Invasive Systeme beeinträchtigen den Fahrer und sind für die Praxis weniger geeignet. 
Aufgrund der beschriebene Vorteile von \acl{BS}, soll das zu entwickelnde System jedoch mit eben diesen arbeiten und das Problem des Tragekomforts gelöst werden.\\

Tragekomfort / Genauigkeit\\
EEG und EKG stehen in der Simulationsumgebung der \acl{RTU} zur Verfügung. Das System soll in der Lage sein, einfallende Müdigkeit zu erkennen und in angemessener Weise zu reagieren. Bei der Anwendung der Sensoren, soll insbesondere auf den Tragekomfort geachtet werden. Im besten Fall nimmt der Fahrer die Sensoren nicht wahr und wird durch diese nicht abgelenkt. Das EEG scheidet bei dieser Betrachtung von vorne herein aus und wird im Test nur zur Verfeinerung bzw. Validierung genutzt. Im Bereich des EKGs existieren Lösungen mit höheren Tragekomfort. Diese liefern jedoch oftmals weniger gute Daten. Daher muss gezeigt werden, dass die Genauigkeit hoch und die Fehlerrate möglichst niedrig bleibt. 
Darum schlagen wir eine Umsetzung mit einem EKG Brustband (Siehe Kapitel \ref{subsec:sim}) vor, welches bei der Entwicklung von einem EEG validiert wird. Ein Austausch des Brustbandes durch einen Pulsmesser oder eine Smartwatch ist vorstellbar.\\

Portabel\\
Tests mit übermüdeten Fahrern im Straßenverkehr lassen sich aus naheliegenden Gründen nicht durchführen. Die Fremd- und Selbstgefährdung ist deutlich zu hoch, auch wenn sich so die realistischsten Ergebnisse erzielen lassen. Auch die Durchführung in einem echten Fahrzeug und einem abgegrenzten Testgelände, würden zumindest ein Risiko für den Fahrer bedeuten. Darum bleibt meist nur eine Simulation, um das System zu \acl{ME} zu testen. 
Dennoch sind Feldversuche ab einem gewissen Entwicklungsstand unumgänglich und sei es nur, um zu Prüfen, ob es in einer realen Testfahrt zu Fehlalarmen kommt. Darum muss die Anwendung sowohl im Simulator, als auch in einem realen Fahrzeug funktionieren und ohne großen Aufwand portiert werden können. Wenn dies gelingt, kann die Anwendung sowohl in unserem Simulator, einem realen Fahrzeug oder einem anderen Simulator genutzt werden, um dort etablierte Systeme zur \acl{ME} zu ergänzen oder zu  validieren.
Das vorgestellte System muss demnach mit möglichst wenig Hardware auskommen und sich leicht auf andere Geräte portieren lassen. Im einfachsten Fall genügt ein Laptop oder Smartphone, einer Erweiterung auf einer Smartwatch ist vorstellbar.\\

Um das System auch zur Verbesserung oder Validierung anderer Systeme zur \acl{ME} zu nutzen, werden öffentliche Schnittstellen definiert und ein sauberes Logging der Daten implementiert. Die Daten der Überwachung sollen mit eindeutigem Zeitstempel versehen werden. So können sie mit anderen Daten, wie Videoaufzeichnung oder die anderer Systeme, zu einem späteren Zeitpunkt verglichen werden. 

Erkennt das System eine drohende Müdigkeit, soll es den Fahrer über ein optisches oder akustisches Signal warnen. Das System soll mehrere Warnstufen zu kennen und je nach Müdigkeitsgrad reagieren. Ebenso wäre ein Rückmeldemechanismus vorstellbar, bei dem der Fahrer dem System einen Fehlalarm mitteilen kann. Diese Erweiterung ist jedoch mit Vorsicht zu genießen, da sich der Fahrer falsch einschätzen kann.

\begin{itemize}
  \item Warum noch eine Arbeit über \acl{ME}?
  \item Warum EKG / EEG?
\end{itemize}


\section{Evaluationsplan}
\label{chap:eval}

Das zu entwickelnde System wird in mehreren Schritten implementiert. In der ersten Phase werden Tests mit den zur Verfügung stehenden \acl{BS} durchgeführt. Diese werden in die Simulationsumgebung der \acl{RTU} integriert und erste Tests am Simulator durchgeführt. In der zweiten Phase soll das getestete System in einem Feldversuch in ein echtes Fahrzeug integriert werden.\\

\subsection{Sensordaten}
Für Phase eins muss zuerst die Hardware (EKG, EEG) an den Daten-Computer angeschlossen werden. Ist die Integration erfolgreich, werden erste Tests mit den Sensoren durchgeführt - Latenz oder Störsignale müssen beachtet werden. 
\begin{itemize}
	\item Tests mit EEG und EKG
\end{itemize}

\subsection{Versuchsaufbau}
Das portable System zu \acl{ME} wird am Fahrsimulator der \acl{RTU} entwickelt und getestet.
Die Simulation kann jedoch nur ein Modell der Wirklichkeit sein und liefert nur eingeschränkte Ergebnisse. So konnten Blana et al. \cite{Blana_1} zeigen, dass sich das Fahrverhalten bei höheren Geschwindigkeiten im echten Straßenverkehr und im Simulationsumfeld unterscheidet. Engstrom et al.  \cite{Engstrom_2322937} konnten ebenfalls Unterschiede feststellen, zeigten jedoch auch, das Tests im Simulator dennoch valide Ergebnisse liefern können. 

\begin{itemize}
	\item Skizze / Bild?
\end{itemize}

\subsection{Testszenario}
Um Daten mit einfallender Müdigkeit zu erhalten, muss ein passendes Szenario im Simulator erstellt werden. Es gilt beim Versuchsaufbau eine möglichst große Chance auf Sekundenschlaf bei den Probanden zu provozieren. Horne und Reyner legten nahe, dass die meisten Unfälle im Zusammenhang mit Schlaf in Großbritannien zwischen 02:00 - 06:00 und 14:00 - 16:00 passierten \cite{Horne_1757738}. Weiterhin lässt sich beobachten, dass Personen die 24 Stunden gar nicht oder nicht ausreichend (< 6h) geschlafen haben, deutlich anfälliger für Sekundenschlaf sind \cite{Peters_1}. 
Ein weiterer Faktor ist das Testszenario selbst. Es sollte möglichst gut erkennbar machen, ob der Proband gerade Fahrfehler macht und diese mit seinem Wachheitsgrad zusammenhängen. Weiterhin kann auch eine "`langweilige Teststrecke"' eine schnellere Ermüdung begünstigen. Langes geradeaus fahren und wenig Abwechslung sind zwar langweilig, aber nicht sehr anstrengend, darum sollte es eine Aufgabe sein, die den Probanden über die ganze Zeit fordert. Auch die Länge der Fahrt spielt eine Rolle, je länger die Fahraufgabe dauert, desto größer die Chance auf das Eintreten einer einfallenden Müdigkeit.


\begin{itemize}
  \item Szenario entwerfen
  \item Hardware einrichten und verdrahten
  \item Daten vom Sensor bis zur Anwendung
  \item Daten aufbereiten und signifikante Features herausarbeiten
  \item Trainingsdaten erstellen
  \item Passenden Machine Learning Algorithmus finden und trainieren
  \item Anwendung Testen
\end{itemize}

\begin{itemize}
  \item Unterschiede / Einschränkungen echtes Fahrzeug / Simulator
  \item Versuchsaufbau
\end{itemize}


\section{Ergebnis}
\label{chap:result}
\begin{itemize}
  \item Ergebnis / Evtl. Prototyp
  \item Portierung der Anwendung von Simulator zu echtem Auto
\end{itemize}

\section{Fazit und Ausblick}
\label{chap:outro}
\begin{itemize}
  \item Weitere Schritte
\end{itemize}


\balance
\bibliographystyle{unsrt} % abbrv, alpha, plain, unsrt, apalike
\bibliography{Quellen,Zotero}


\end{document}
