\documentclass[8pt,a5paper]{acm_proc_article-sp}
% fix umlauts
\usepackage[ngerman]{babel}
\usepackage[utf8]{inputenc} 
\usepackage[T1]{fontenc}  % Times new Roman
\usepackage{mathptmx}
\usepackage[]{blindtext}
\usepackage[a5paper, left=1.70cm, right=1.20cm, top=1.30cm, bottom=1.40cm]{geometry}
\usepackage{epstopdf}
% balance columns. original from acm doesn't work
\usepackage{balance}
% colors
\usepackage[compact]{titlesec}
\usepackage{natbib}
\titlespacing{\section}{0pt}{*0.7}{*0.5}

\usepackage[nolist]{acronym}
 
\usepackage[usenames,dvipsnames]{xcolor}
% automatic crosslinks
\usepackage[colorlinks=true,citecolor=black,linkcolor=black,filecolor=black,urlcolor=black,pagebackref=true]{hyperref}
\hypersetup{colorlinks=true,citecolor=black,linkcolor=black,filecolor=black,urlcolor=black,pagebackref=true}
%glossaries
%\usepackage{makeidx}
%\makeindex
%\usepackage[nomain]{glossaries}
%\makeglossaries
%\makeindex

\newcommand{\glspl}[1]{{#1}}

% http://en.wikibooks.org/wiki/LaTeX/Glossary
% http://mirror.informatik.uni-mannheim.de/pub/mirrors/tex-archive/macros/latex/contrib/glossaries/glossariesbegin.pdf

% Definitionsliste
\newcommand{\defitem}[1]{\item[#1]\phantomsection\label{#1}\hfill\\} 
\newcommand{\defref}[1]{\hyperref[#1]{#1}} 
\newcommand{\rem}[1]{}
% Marking colors
\definecolor{todo}{rgb}{1,0.2,0.2}
\definecolor{reconsider}{rgb}{0.6,0.6,0.3}
\newcommand{\todo}[1]{{\color{todo} #1}}
\newcommand{\reconsider}[1]{{\color{reconsider} #1}}



\graphicspath{{images/}{./}}

\title{Aktuelle Anforderungen an Fahrerassistenzsysteme \titlenote{ \scriptsize \flushleft Betreuer Hochschule:  \  \ Prof. Dr. Martinez\\ \qquad \qquad \qquad \qquad \quad \ \  Hochschule Reutlingen\\ \qquad \quad \quad \quad \qquad \qquad \ \ Natividad.Martinez@Reutlingen-\\ \qquad \qquad \qquad \qquad \quad \ \ University.de\\  Informatics Inside 2015 II\\ Wissenschaftliche Vertiefungskonferenz \\ 18. November 2015, Hochschule Reutlingen\\  \copyright 2015 Paul Pasler}}


\numberofauthors{3}
\author{
	\alignauthor
	  \center
		\aufnt{Paul Pasler}\\
          \affaddr{Reutlingen University}\\
        \textbf{\textsf{Paul.Pasler@Student.Reutlingen-University.DE}}
}



\begin{document}

\begin{acronym}
\acro{FAS}{Fahrerassistenzsystem}
\acro{FASs}{Fahrerassistenzsysteme}
\acro{ME}{Müdigkeitserkennung}
\acro{ADAS}{Advanced Driver Assistance System}
\acro{ADASs}{Advanced Driver Assistance Systems}
\end{acronym}


\maketitle
\sloppypar{
\begin{abstract}
In dieser Ausarbeitung gibt es einen kleinen Überblick zu \acl{FAS}en. Weiterhin wird die Funktionsweise und verschiedene Umsetzungen von Systemen zur \acl{ME} vorgestellt.
Der Ansatz mit Body-Sensorik wird auf seine Umsetzbarkeit im Simulationsumfeld der Reutlingen University evaluiert.

\end{abstract}

\keywords{
\acf{ADAS}, \acl{FAS}, \acl{ME}
}

\category{A.0}{ACM}{
sein eigenes offizielles Klassifizierungssystem. Die komplette Liste dieser Kategorien finden Sie unter dem folgenden Link:
\href{URL}{http://www.acm.org/about/class/1998/}
}

\section{Einleitung}
\label{chap:intro}


\subsection{\acl{FASs}}
\label{sec:fas}
 
Fristeten \acl{FASs} vor wenigen Jahren ein Nieschendasein in Oberklassewage, werden sie immer günstiger und beliebter. So halten sie auch in Mittel- und Kleinwagen Einzug.

\begin{itemize}
  \item Überblick und Klassifizierung   \cite{Bertoldi:2010:MAD:2002368.2002370} \cite{Feld:2010:MIA:1719970.1720063}
  \item Aktuelle Entwicklungen / Stand der Technik
\end{itemize}

\subsection{Simulationsumgebung}
\label{sec:sim}
\begin{itemize}
  \item Aufbau des Simulators
  \item Technische Kommunikation im Fahrzeug / Simulator \cite{serial}
\end{itemize}



\section{Systeme zur \acl{ME}}
\label{chap:me}

Die \acl{ME} merkt an Hand verschiedener Daten, ob der Fahrer gerade Müde wird und empfiehlt eine Pause.

\begin{itemize}
  \item Funktion
  \item Vergleich verschiedener Vorgehen 
  \begin{itemize}
    \item ...using skin conductance and oximetry pulse \cite{Bundele:2009:DFV:1806338.1806478}
    \item ... Pulse Wave by Photoplethysmography Signal Processing \cite{Park:2009:DDD:1667780.1667798}
    \item using automatic visual analysis \cite{Haloi:2014:CDB:2662117.2662126}
    \item Use of EEG for Validation of Flicker-Fusion Test \cite{Ronzhina:2011:UEV:2093698.2093733}
    \item Synchronising Physiological and Behavioural Sensors in a Driving Simulator \cite{Taib:2014:SPB:2663204.2663262}
    \item Driver's drowsiness inference based on hidden Markov model using head pose and eye-blink tracking (leider nur auf koreanisch) \cite{Choi:2014:DDI:2729485.2729524}
    
    
\end{itemize} (Computervision, Body Sensorik, Mustererkennung...)
\end{itemize}

\section{Evaluation einer \acl{ME} im Simulationsumfeld}
\label{chap:eval}

\begin{itemize}
  \item Unterschiede / Einschränkungen echtes Fahrzeug / Simulator
  \item Versuchsaufbau
  \item Ergebnis / Evtl. Prototyp
\end{itemize}

\subsection{Fazit}
\label{chap:outro}
\begin{itemize}
  \item Weitere Schritte
\end{itemize}


\balance
\bibliographystyle{abbrv} % abbrv, alpha, plain, unsrt, apalike
\bibliography{Quellen,Zotero}


\end{document}
